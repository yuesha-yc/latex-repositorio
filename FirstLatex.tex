\documentclass[12pt, letterpaper]{article}
\usepackage[utf8]{inputenc}
\usepackage{graphicx}

\title{First \LaTeX \space Document}
\author{Yichen Wang \thanks{Powered by \LaTeX{}}}
\date{April 2021}
\graphicspath{ {images/} }

\begin{document}

\maketitle

\tableofcontents

\begin{abstract}
Texpad is a very great editor. Texpad is a very great editor. Texpad is a very great editor.  Texpad is a very great editor.  Texpad is a very great editor.  Texpad is a very great editor.  Texpad is a very great editor.  Texpad is a very great editor.  Texpad is a very great editor.  Texpad is a very great editor.  Texpad is a very great editor.  Texpad is a very great editor.  Texpad is a very great editor.  Texpad is a very great editor.  Texpad is a very great editor.   

Everyone is good Including me
\end{abstract}

% This line is commented

Hello
World
H
e
l
l \newline world \newline world

New world

\part{The Forewords}

Some of the \textbf{greatest} 
discoveries in \textit{science} 
were made by \underline{\textbf{accident}}.

Some of the greatest \emph{discoveries} 
in science 
were made by accident.

\textit{Some of the greatest \emph{discoveries} 
in science 
were made by accident.}

\textbf{Some of the greatest \emph{discoveries} 
in science 
were made by accident.}

\begin{figure}[h]
    \centering
    \includegraphics[width=0.25\textwidth]{grass}
    \caption{a nice grass}
    \label{fig:grass1}
\end{figure}

As you can see in the figure \ref{fig:grass1}, the 
function grows near 0. Also, in the page \pageref{fig:mesh1} 
is the same example.

\begin{itemize}
  \item This is a Jay.
  \item This is a Kevin.
\end{itemize}

\begin{enumerate}
	\item CA
	\item TN
	\item WA
\end{enumerate}

\part{Elegance}

%\chapter{First Chapter}

\section{Physics}

\subsection{Special Relativity}

\subsubsection{Energy Mass Equivalence}

In physics, the mass-energy equivalence is stated 
by the equation $E=mc^2$, discovered in 1905 by Albert Einstein.

The mass-energy equivalence is described by the famous equation
\[ E=mc^2 \]
discovered in 1905 by Albert Einstein. 
In natural units ($c = 1$), the formula expresses the identity
\begin{equation}
E=m
\end{equation}

\section{Math}

Subscripts in math mode are written as $a_b$ and superscripts are written as $a^b$. These can be combined an nested to write expressions such as

\[ T^{i_1 i_2 \dots i_p}_{j_1 j_2 \dots j_q} = T(x^{i_1},\dots,x^{i_p},e_{j_1},\dots,e_{j_q}) \]
 
We write integrals using $\int$ and fractions using $\frac{a}{b}$. Limits are placed on integrals using superscripts and subscripts:

\[ \int_0^1 \frac{dx}{e^x} =  \frac{e-1}{e} \]

Lower case Greek letters are written as $\omega$ $\delta$ etc. while upper case Greek letters are written as $\Omega$ $\Delta$.

Mathematical operators are prefixed with a backslash as $\sin(\beta)$, $\cos(\alpha)$, $\log(x)$ etc.

\section{Tables}

\subsection{Simple Table}

\begin{center}
\begin{tabular}{ c c c }
 cell1 & cell2 & cell3 \\ 
 cell4 & cell5 & cell6 \\  
 cell7 & cell8 & cell9    
\end{tabular}
\end{center}

\subsection{Border Table}

\subsubsection{Ex1}
\begin{center}
\begin{tabular}{ |c|c|c| } 
 \hline
 cell1 & cell2 & cell3 \\ 
 cell4 & cell5 & cell6 \\ 
 cell7 & cell8 & cell9 \\ 
 \hline
\end{tabular}
\end{center}

\subsubsection{Ex2}
\begin{center}
 \begin{tabular}{||c c c c||} 
 \hline
 Col1 & Col2 & Col2 & Col3 \\ [0.5ex] 
 \hline\hline
 1 & 6 & 87837 & 787 \\ 
 \hline
 2 & 7 & 78 & 5415 \\
 \hline
 3 & 545 & 778 & 7507 \\
 \hline
 4 & 545 & 18744 & 7560 \\
 \hline
 5 & 88 & 788 & 6344 \\ [1ex] 
 \hline
\end{tabular}
\end{center}

\subsection{Labels, captions, and references}

Table \ref{table:data} is an example of referenced \LaTeX{} elements.

\begin{table}[h!]
\centering
\begin{tabular}{||c c c c||} 
 \hline
 Col1 & Col2 & Col2 & Col3 \\ [0.5ex] 
 \hline\hline
 1 & 6 & 87837 & 787 \\ 
 2 & 7 & 78 & 5415 \\
 3 & 545 & 778 & 7507 \\
 4 & 545 & 18744 & 7560 \\
 5 & 88 & 788 & 6344 \\ [1ex] 
 \hline
\end{tabular}
\caption{Table to test captions and labels}
\label{table:data}
\end{table}



\end{document}
